\documentclass[11pt,a4paper]{article}
\usepackage[utf8]{inputenc}
\usepackage[T1]{fontenc}

\textheight=22cm
\textwidth=14.945cm
\topmargin=0.5cm
\headheight=0cm
\headsep=0cm
\oddsidemargin=0.5cm
\evensidemargin=0.5cm
\columnsep=0.8cm
\renewcommand{\baselinestretch}{1.3}

\begin{document}

\begin{titlepage}
  \begin{center} \sloppy
    \large \textsc{Huffman-coding - a data structures project}
    \vfill

    \huge \textbf{Design document} \vfill

    \LARGE Jussi Mäki - joamaki@gmail.com
    \vspace{3mm}

    Data structures project\\

    \vfill

    Helsinki \today
    % \number\day .\number\month .\number\year

  \end{center}
\end{titlepage}

\section {Introduction}

This document describes the design of data structures project
involving the use of the Huffman-coding algorithm.

The Huffman code algorithm is a greedy algorithm that takes a list of
characters and corresponding frequencies in input data and outputs a
tree which describes the optimal prefix code for each character. This
resulting prefix code tree can be then used to compress the input data
and to decompress it.

The program implemented in this project shall provide
compression and decompression of arbitary data using
the Huffman code algorithm.

\section {Data structures}

The huffman code algorithm as described in \cite{CLRS} requires the
use of an auxilliary data structure to extract the lowest frequency
object, ie. a form priority queue. A binary heap data structure is
well suited for this task and so shall be used as the auxilliary data
structure for the algorithm.

To build the set of character frequencies in the input data a table of
integers the size of possible character values is used.  This table is
then rebuilt as a heap (using frequency as the key) before passing it
to the huffman code algorithm.

\section {Program usage}

The program is controlled by the use of command-line arguments. It
has two modes of operation: compression and decompression.

If no command-line arguments are given it'll default to compressing
the standard input to the standard output. If one filename is given it
will compress and save the output to filename.hc.  Decompression mode
is enabled with the option '-d'. If a filename is given when
decompression mode is enabled and it has the suffix .hc the result is
output to a file with the suffix removed.

Example of compression and decompression:

\begin{verbatim}
$ hc thesis.txt
Compressing thesis.txt to thesis.txt.hc...
$ hc -d thesis.txt.hc
Decompressing thesis.txt.hc to thesis.txt...
\end{verbatim}

\bibliographystyle{plain}
\bibliography{ref}

\end{document}